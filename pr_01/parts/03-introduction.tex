\chapter{Введение}

Практикум посвящен освоению принципов работы вычислительного комплекса Тераграф и получению практических навыков решения задач обработки множеств на основе гетерогенной вычислительной структуры.
Участникам предоставляется доступ к удаленному серверу с ускорительной картой и настроенными средствами сборки проектов, конфигурационный файл для двухъядерной версии микропроцессора Леонар Эйлер, а также библиотека leonhard x64 xrt c открытым исходным кодом.

Целью работы является разработка программы для хост--подсистемы и обработчики программного ядра, выполняющие действия по варианту.

Для \textit{15 варианта} необходимо разработать \textbf{устройство интегрирования}.
То есть сформировать в хост--подсистеме и передать в SPE 256 записей с ключами $x$ и значениями $f(x)=x^2$ в диапазоне значений $x$ от 0 до 1048576. Передать в sw\_kernel числа $x1$ и $x2$ ($x2 > x1$).
В хост--подсистему вернуть сумму значений $f(x)$ на диапазоне $(x1, x2)$.
Сравнить результат с ожидаемым.