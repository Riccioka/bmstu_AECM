\chapter{Задания} 
\section{Задание 1}

Содержимое файла \textit{task.s}, который соответсвует \textbf{варианту 15}:
\begin{lstlisting}
    .section .text
    .globl _start;
    len = 9 #Размер массива
    enroll = 1 #Количество обрабатываемых элементов за одну итерацию
    elem_sz = 4 #Размер одного элемента массива

_start:
    la x1, _x
    addi x20, x0, (len-1)/enroll
    lw x31, 0(x1)
    addi x1, x1, elem_sz*1
lp:
    lw x2, 0(x1)
    bltu x2, x31, lt
    add x31, x0, x2 #!
lt:
    add x1, x1, elem_sz*enroll
    addi x20, x20, -1
    bne x20, x0, lp
lp2: j lp2

    .section .data
_x:     .4byte 0x1
    .4byte 0x2
    .4byte 0x3
    .4byte 0x4
    .4byte 0x5
    .4byte 0x6
    .4byte 0x7
    .4byte 0x8
    .4byte 0x9

\end{lstlisting}

Псевдокод:
\begin{lstlisting}{language=C}

    #define len 9
    #define enroll 1
    #define elem_sz 4

    int _x[]={1,2,3,4,5,6,7,8,9};

    void _start() {
        int x20 = (len-1)/enroll;
        int *x1 = _x;
        int x31 = x1[0];
        x1++;
    
        do {
            int x2 = x1[0];
            if (x2 >= x31)
                x31 = x2; // #
            
            x1++;
            x20--;
        } while(x20 != 0);

        while(1){}
    }

\end{lstlisting}

Результат выполнения компиляции:
\begin{lstlisting}{language=bash}
    SYMBOL TABLE:
    80000000 l    d  .text 00000000 .text
    80000030 l    d  .data 00000000 .data
    00000000 l    df *ABS* 00000000 task.o
    00000009 l       *ABS* 00000000 len
    00000001 l       *ABS* 00000000 enroll
    00000004 l       *ABS* 00000000 elem_sz
    80000030 l       .data 00000000 _x
    80000014 l       .text 00000000 lp
    80000020 l       .text 00000000 lt
    8000002c l       .text 00000000 lp2
    80000000 g       .text 00000000 _start
    80000054 g       .data 00000000 _end
    
    
    
    Дизассемблирование раздела .text:
    
    80000000 <_start>:
    80000000: 00000097           auipc x1,0x0
    80000004: 03008093           addi x1,x1,48 # 80000030 <_x>
    80000008: 00800a13           addi x20,x0,8
    8000000c: 0000af83           lw x31,0(x1)
    80000010: 00408093           addi x1,x1,4
    
    80000014 <lp>:
    80000014: 0000a103           lw x2,0(x1)
    80000018: 01f16463           bltu x2,x31,80000020 <lt>
    8000001c: 00200fb3           add x31,x0,x2
    
    80000020 <lt>:
    80000020: 00408093           addi x1,x1,4
    80000024: fffa0a13           addi x20,x20,-1
    80000028: fe0a16e3           bne x20,x0,80000014 <lp>
    
    8000002c <lp2>:
    8000002c: 0000006f           jal x0,8000002c <lp2>
    
    Дизассемблирование раздела .data:
    
    80000030 <_x>:
    80000030: 0001                 c.addi x0,0
    80000032: 0000                 c.unimp
    80000034: 0002                 c.slli64 x0
    80000036: 0000                 c.unimp
    80000038: 00000003           lb x0,0(x0) # 0 <enroll-0x1>
    8000003c: 0004                 0x4
    8000003e: 0000                 c.unimp
    80000040: 0005                 c.addi x0,1
    80000042: 0000                 c.unimp
    80000044: 0006                 c.slli x0,0x1
    80000046: 0000                 c.unimp
    80000048: 00000007           0x7
    8000004c: 0008                 0x8
    8000004e: 0000                 c.unimp
    80000050: 0009                 c.addi x0,2

\end{lstlisting}

Содержимое файла \textit{task.hex:}
\begin{lstlisting}
    00000097
    03008093
    00800a13
    0000af83
    00408093
    0000a103
    01f16463
    00200fb3
    00408093
    fffa0a13
    fe0a16e3
    0000006f
    00000001
    00000002
    00000003
    00000004
    00000005
    00000006
    00000007
    00000008
    00000009    
\end{lstlisting}

В регистре x31 в конце выполнения программы должно содержаться максимальное значение массива. В нашем случае: \textbf{9}.
